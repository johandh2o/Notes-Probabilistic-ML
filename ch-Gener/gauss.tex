%%%%%%%%%%%%%%%%%%%%%%%%%%%%%%%%%%%%%%%%%%%%%%%%%%%%%%%%
%------------- SECTION ---------------------------------
%%%%%%%%%%%%%%%%%%%%%%%%%%%%%%%%%%%%%%%%%%%%%%%%%%%%%%%%

\section{Introduction}

The \red{multivariate normal distribution (MVN)} is the most widely used joint probability distribution for continuous variables. This follows from the \red{Central limit theorem} and the \red{Maximum entropy theorem}. The first one establishes that, in some situations, when independent random variables are added, their properly normalized sum tends toward a normal distribution. The second one establishes that, among continuous distributions with specified mean and covariance, the normal distribution has maximum entropy. This implies that the first two moments are usually all that we need to estimate from data, making as few additional assumptions as possible.

%------------- SUBSECTION ---------------------------------
\subsection{MLE for MVN}
Let $X_i$ be a random variable with $\supp{X_i}\in\R^d$, and $X_i\dist{iid}\text{Nor}(\mu,\Sigma)$, with $\mu\in\R^d$, $\Sigma\in\R^{d\times d}$, $\Sigma\succeq 0$, then it has PDF:
\begin{equation*}
    f(x|\mu,\Sigma) = \frac{1}{(2\pi)^{d/2}\abs{\Sigma}^{1/2}}\exp[-\frac{1}{2}\tp{(x-\mu)}\inv{\Sigma}(x-\mu) ]
\end{equation*}

With a random sample of size $n$, its moments can be estimated by MLE:
\begin{align*}
    \hat{\mu} &= \frac{1}{N}\sum_{i=1}^nx_i = \bar{x}\\
    \hat{\Sigma} &= \frac{1}{N}\sum_{i=1}^n (x_i-\hat{\mu})\tp{(x_i-\hat{\mu})} = \frac{1}{N}\sum_{i=1}^n x_i\tp{x_i}-\hat{\mu}\tp{\hat{\mu}}
\end{align*}

%%%%%%%%%%%%%%%%%%%%%%%%%%%%%%%%%%%%%%%%%%%%%%%%%%%%%%%%
%------------- SECTION ---------------------------------
%%%%%%%%%%%%%%%%%%%%%%%%%%%%%%%%%%%%%%%%%%%%%%%%%%%%%%%%

\section{Discriminant Analysis}

One important application of MVNs is to define the class conditional densities in a generative classifier such that $X_i|(Y_i=C_k)\dist{iid}\text{Nor}(\mu_k,\Sigma_k)$\footnote{This model reduces to the Naïve Bayes model when $\Sigma_k$ is diagonal for every class $k$}. Under these assumptions, the predictive posterior probability of $Y=C_k,\ k\in\{1,...,K\}$ given $X=x\in\R^d$ would be:
\begin{equation*}
    p(C_k|x,\mu,\Sigma,\delta) = \frac{p(C_k|\delta_k)f(x|C_k,\mu_k,\Sigma_k)}{\sum_{c=1}^K p(C_c|\delta_c)f(x|C_c,\mu_c,\Sigma_c)} = \frac{\delta_k\abs{\Sigma_k}^{-1/2} \exp[\frac{-1}{2}\tp{(x-\mu_k)}\inv{\Sigma_k}(x-\mu_k)]}{\sum_{c=1}^K \delta_c\abs{\Sigma_c}^{-1/2} \exp[\frac{-1}{2}\tp{(x-\mu_c)}\inv{\Sigma_c}(x-\mu_c)]}
\end{equation*}

It is to notice that $\ln p(C_k|x,\mu,\Sigma,\delta)$ is a quadratic function of $x$ for fixed values of the parameters: the decision boundaries look like quadratic functions in 2D. For this, the model is called \red{Quadratic Discriminant Analysis (QDA)}.

Class prediction is achieved by:
\begin{equation*}
    \hat{y}(x) = C_{\varkappa},\ \varkappa \in \argmax_{k\in \{1,\dots,K\}} p(C_k|x,\mu,\Sigma,\delta) = \argmax_{k\in \{1,\dots,K\}} \delta_k\abs{\Sigma_k}^{-1/2} \exp[\frac{-1}{2}\tp{(x-\mu_k)}\inv{\Sigma_k}(x-\mu_k)]
\end{equation*}

This can be thought of as a \red{nearest centroids classifier} using the \red{Mahalanobis distance}.